\documentclass[12pt]{article}

%%% Nützliche Pakete
\usepackage{graphicx}
\usepackage[T1]{fontenc}
\usepackage[utf8]{inputenc}
\usepackage[english]{babel}
\usepackage{listings}
\usepackage{xcolor}
\usepackage{eso-pic}
\usepackage{mathrsfs}
\usepackage{url}
\usepackage{amssymb}
\usepackage{amsmath}
\usepackage{multirow}
\usepackage{hyperref}
\usepackage{booktabs}
\usepackage{bbm}
\usepackage{longtable}
\usepackage[top=2.5cm,bottom=2.5cm,left=2.5cm,right=2.5cm]{geometry}
\usepackage{lmodern}
\renewcommand*\familydefault{\sfdefault}

\input{Definitions.tex}

%%% Zeilenabstand
\linespread{1.3}

%%% Für Bibliographie
	\usepackage[
		backend=bibtex,
		minbibnames=3, 
		bibstyle=authoryear, 
		citestyle=authoryear, 
		natbib,
		sorting=nyt,
		firstinits=true,
		maxbibnames=99,
		maxcitenames = 2
		]{biblatex}

	\renewbibmacro{in:}{%
	  \ifentrytype{article}{}{\printtext{\bibstring{in}\intitlepunct}}}

	 \DeclareFieldFormat*{citetitle}{#1}
	 \DeclareFieldFormat*{title}{#1}
	 \DeclareNameAlias{sortname}{last-first}
	%%% Füge Bibliographie zur Arbeit hinzu
	\addbibresource{ref1.bib}
	
%%% Bis wieviele Unter-Sektionen soll nummeriert werden	
	\setcounter{secnumdepth}{4}
	\setcounter{tocdepth}{4}
%%% maybe make citation bold

\title{The Bootstrap - Numerical Procedures and Applications}
\author{Erin Sprünken}
\date{31.11.2020}

\pagenumbering{roman}
\begin{document}

		%%%% Titelseite %%%%
		
		\begin{titlepage}
		\pagestyle{empty}
		\begin{center}

		    {\Large{\bf The Bootstrap - Numerical Procedures and Applications}} \vspace{0.5cm}


		    {\normalsize Seminar Paper submitted\\\vspace{0.5cm}
		    to}\\\vspace{0.5cm}
		    {\normalsize{\bf
			   	 Prof. Dr. Brenda López-Cabrera 
		 	     }}\\\vspace{0.5cm}
		    {\normalsize Humboldt-Universit\"at zu Berlin \\
		    School of Business and Economics \\
		    Climate, Weather and Energy Analysis} \vspace{1cm}


		    {\normalsize by \\\vspace{0.5cm}
		    {\bf Erin Sprünken} \\
		    (581608)} \vspace{1cm}


		    {\normalsize in partial fulfillment of the requirements \\
		    for the Seminar \\
		    %%% ggf. hier Master of Science (M.Sc.) einfügen!!
		    {\bf Numerical Introductory Course} \\ 
		    Berlin, \today} %%% Statt today hier ggf. ein fixes Datum einfügen!
		    %\vfill
		    
		    %%%% Logo der Universität und falls gewünscht des Instituts!
		    \begin{figure}[!b]
		    \centering
		    %%%% Erstes Logo!
		    \begin{minipage}{0.45\textwidth}
		        \centering
		        \includegraphics[scale = 0.9]{hulogo.pdf} % first 		figure itself
		    \end{minipage}\hfill
		    %%% Hier kommt das zweite Logo, falls gewünscht die Prozentzeichen für Kommentare entfernen!
%		    \begin{minipage}{0.45\textwidth}
%		       \centering
%		        \includegraphics[scale = 0.45]{irtglogo1.pdf} % second figure itself
%		    \end{minipage}
		\end{figure}

\end{center}
\end{titlepage}

%%% Ende der Titelseite

%%%
\newpage
\pagestyle{plain}
\pagenumbering{Roman}

\tableofcontents

\newpage

\listoftables

\newpage

\listoffigures

\newpage

\pagenumbering{arabic}

%%% Hier die einzelnen Section-Tex-Files einfügen! Natürlich kann man den Section-Command auch in die einzelnen Files legen, aber so ist etwas übersichtlicher!
\section{Motivation}
motivation

* basics for master thesis in ranking procedures
* speed of c/c++ based implementation => basically interest in lower level language application/simulation
*
\clearpage

\section{Related Literature}
In this section we provide an overview of related literature to the topic of the bootstrap. Important to mention is the article of \citet{boot1} who generalised the jackknife principle to what is the classical bootstrap today. Since then, many advancements, generalizations and applications were conducted by numerous authors. \citet{prob1} provides a general overview of resampling techniques in chapter 13 and \citet{stat1} describe and relate bootstrap methods to Maximum Likelihood, Bayesian Inference and Machine Learning techniques in chapter 8. \\
\\
\noindent \citet{boot4} find that bootstrap and permutation methods for matched pairs give valid asymptotic results even for different distributions which have no treatment effect under the null hypothesis. Their paper relates to estimation of t-type test statistics for paired samples and overcomes the problems of small sample sizes, heterogeneity and different distributions. \citet{boot2} find valid results applying the so-called wild bootstrap method to problems of repeated measurements. The nonparametric bootstrap approach delivers better results than WALD or ANOVA statistics, which are especially useful for small sample sizes. \citet{boot3} find that wild-bootstrapping ranked-based procedures overcome assumptions about normality or homogeneity in general factorial measure designs and deliver asymptotically correct multiple contrast tests. Furthermore, \citet{boot5} find that parametric bootstrap can, under minimal assumptions, overcome the problems of normality assumptions or equal covariance matrices in multivariate factorial designs. \\
\\
\noindent \citet{boot6} and \citet{boot7} describe bootstrap methods for timeseries, which again are used in applied research. For example, \citet{alla} refer to the latter one cominbing portfolios. \citet{erin} use a parametric bootstrap approach to numerically compute optimal CVaR (Conditional Value-at-Risk) portfolios.


\clearpage

\section{Applications}
As already mentioned before, various applications for the bootstrap method exist. For example, early clinical trials usually suffer from small sample sizes. For example, consider a very early stage for a new drug. Since it's effects are only considered theoretically, ethical standards would forbid testing such a drug on a large sample of patients. On the other hand, small samples are not reliable when it comes to statistical inference. However, valid inference is necessary to judge about approving or declining the public usage of this treatment. The solution lies in bootstrapping a small sample. For example, the clinical trial consists of 16 patients receiving the new treatment or the placebo, this leads to a two-sample of matched pairs design. Here, bootstrap solves this small sample problem and leads to reliable results, see \citet{boot4} and \citet{boot2}.\\
\\
\noindent In finance, especially portfolio management, parametric bootstrap and resampling in general can provide large samples when analytic solutions are not possible. Consider the following minimization problem (Condition Value-at-Risk):
\begin{align}
	\min_{x} \quad - \frac{1}{1 - \alpha}\int_{x^{\intercal}\mu \leq -VaR_{\alpha}(x)} x^{\intercal}\mu f\left(x^{\intercal}\mu \mid x\right)\dd x^{\intercal}\mu.
\end{align}
This has no analytic solutions. However, a portfolio manager wants to obtain the portfolio weights $x$. \citet{erin} solve this problem by sampling a vector of weights $x$ from the uniform distribution $n_{b}$ times and compute $n_{b}$ (empirical) CVaRs and choose the vector $x$ of weights which corresponds to the smallest CVaR of all.\\
\\
\noindent Bagging, as mentioned and described in \citet{stat1}, uses bootstrapping to improve Machine Learning techniques. By bootstrapping and its property of drawing with replacement, one creates $n_b$ samples on which an algorithm can be trained and estimated. The $n_b$ models are then aggregated into one (for example by majority vote), see \citet{rf2}. This parallel design is distinct to sequential designs such as Boosting. Random Forests are a special case of this, since it grows $n_b$ trees instead of only one and thus significantly reduces the variance of the tree model, see \citet{rf1} and \citet{rf3}.\\
\\
\noindent Generally, bootstrapping can be applied to many fields of statistical and applied sciences. 
\clearpage

\section{Methodology}
This section will cover formal aspects of the bootstrapping and is divided into two parts. The first one is a short overview of advantages and disadvantages. The second part will briefly describe mathematical aspects of the different applications covered within this seminar paper. 

\subsection{Advantages and Disadvantages}

\subsection{Mathematical Aspects}
	\subsubsection*{Summary Statistics}
	\subsubsection*{t-Type Test Statistics}
	\subsubsection*{Regression Coefficients}
	

\clearpage

\section{Simulation Settings}\label{sec:sim}
%simulation setting
In this section we present the simulation settings we use to demonstrate bootstrap techniques. For all different problems we adress, $25$ combinations of bootstrapping iterations ($nboot$) and sample sizes ($n$) are computed: $nboot = \{10, 100, 500, 1000, 10000\} \times n = \{5, 10, 50, 100, 200\}$. When it comes to measuring the speed of the implemented algorithm, we use microbenchmarking, calling the function with specified parameters $100$ times. Furthermore, the following problem-specific simulation settings are demonstrated within this paper.

\subsection{Summary Statistics}
Computing the summary statistics is mainly to demonstrate the accuracy of bootstrap techniques. This is, how precisely can we estimate paramters such as mean, median or standard deviation from a sample using bootstrap. To measure this, we compute the MSE and MAE of bootstrap estimates for randomly drawn samples from a $N(0,1$ population for each of the above named combinations. Recall the definitions of MSE and MAE:
\begin{align*}
	MSE \quad &:= \quad \frac{1}{nboot} \sum_{i=1}^{nboot} (\hat{\vartheta}_i - \vartheta)^2 \\
	MAE \quad &:= \quad \frac{1}{nboot} \sum_{i=1}^{nboot} \lvert \hat{\vartheta}_i - \vartheta \rvert,
\end{align*}
where $\vartheta$ is the parameter of interest. We acknowledge, that the choice of $N(0,1)$ population is arbitrary and could be replaced by any other and is only motivated by it's simplicity and popularity.

\subsection{Linear Regression}
In order to demonstrate computational complexity of bootstrap techniques and advantages of lower-level languages such as C++ over high-level languages such as R we examine a simple linear regression problem with four variables from a pre-specified data generating process (DGP). Furthermore, we define two accuracy measures which will be computed for an arbitrary setting, but since accuracy is not the main interest of this part of the paper it will not be done for the whole set of combinations. Let $B  = \{\beta_1, ..., \beta_i, ..., \beta_k\}$, then define the Compund MSE and MAE as:
\begin{align*}
cMSE \quad &:= \quad \frac{1}{\lvert B \rvert} \sum_{i = 1}^{\lvert B \rvert} \frac{1}{nboot} \sum_{j = 1}^{nboot} (\hat{\beta}_{ij} - \beta_i)^2 \\
cMAE \quad &:= \quad \frac{1}{\lvert B \rvert} \sum_{i = 1}^{\lvert B \rvert} \frac{1}{nboot} \sum_{j = 1}^{nboot} \lvert (\hat{\beta}_{ij} - \beta_i) \rvert.
\end{align*}
Further, we defined an arbitrary DGP with the following parameters and data: $X_1$: Discrete uniform from $30$ to $100$, $X_2$: U(50, 230), $X_3$: log(N(100, 7.5)), $X_4$: N(100, 7.5) and \begin{align*}
\beta \quad  &= \quad (-16.046722, -9.140887, 10.661128, -11.014303)
\end{align*}. Thus, we have: $Y = X\beta + \epsilon$, where $\epsilon_i \thicksim N(0,1)$. Of course, specific bootstrapping techniques could work for more specific problems such as heteroskedastic errors or different data settings, but since we want to measure computational complexity and work on different problems in this paper we leave this for further research.

\subsection{t-Test}
Finally, we want to adress the application of bootstrap techniques in statistical test problems. For this, we extensively simulate the t-Test. We note, that the research on statistical tests is much larger, but due to it's extensive application in empirical research the t-Test works well to show properties of bootstrap. However, we definitely recommend further research on other test-problems. Our simulation uses both one- and two-sample problems and, as before, both nonparametric and wild bootstrap. To simulate the Type-I-Error we create four different populations and test against their expectation ($H_0: \mu = \mu_0$) and ignore the required normality to gain some insight about how well the test reaches it's significance level, which is interesting for practioneers. Our actual populations are $N(0,1), Pois(5), Exp(3)$ and $\chi^2(2)$. When it comes to Type-II-Error (or Power, which is $1-\beta$ and thus directly related to it), we only use $N(\mu, \sigma)$ under the alternative and vary the true mean of the population. In the two-sample case we vary the difference in means for the power study. 
\clearpage

\section{Empirical Analysis}
This section covers the empirical results using the methods mentioned earlier. Our interest lies in the aspects of accuracy, computation time and complexity, and for tests the errors of type I and II. 
\subsection{Accuracy}

\subsection{Type I and II Errors}

\subsection{Computational Complexity}
\clearpage

\section{Conclusion}
%conclusion
Finally, we want to draw a conclusion. In our empirical section we found what was expected from a theoretical point of view. Bootstrap techniques can accurately estimate parameters and statistics even under small sample sizes. Furthermore, an extensive simulation of testing the mean showed that bootstrap-versions of the t-Test converge fast to the significance level in terms of Type-I-Error, although for some populations this took longer than for others, dependent on the sample sizes. This highlights one of the main drawbacks of bootstrap in general, namely that the sample itself has to be at least a littlebit representative in order to obtain proper bootstrap results. Nonetheless, with increasing sample sizes and increasing bootstrap iterations the significance level is reached. Also, the power of the t-Test was roughly the same in it's bootstrap versions as it was in it's actual version for nearly all situations. Both aspects demonstrate that bootstrapping tests can be a proper method in practice. Also, bootstrap is more flexible in practice in the sense that one has not to rely on the theoretical distribution function of the test statistic, since this is created with an empirical distribution function (which is again, in theory, converging almost sure to the actual distribution). With a simple regression problem we demonstrated computational complexity of bootstrap and applied different language implementations. Even for such a simple problem the computation time was quite high and increased in this specific case exponentially with the number of bootstrap iterations. As expected, the implementation in a lower-level language, namely C++, provided a huge advantage against R, being roughly $8$ times faster on median time. Although the time was still below a single second, this can be generalized since there exist many problems requiring much more computation time than a simple linear regression estimation. Since this paper only provided a basic overview, we highly recommend a deeper look on topics such as different versions of the wild bootstrap, since we only applied the Rademacher-version, as well as parametric bootstraps for specific problems. Also, many test-problems are of interest, for example multiple contrast tests, F-Test, Anderson-Darling, Kolmogorov-Smirnov or rank-based tests such as Mann-Whitney-U test and other effect-size tests. Even though there is a lot of research in that field, current problems in applied fields require new statistical methods as well. For example, the ongoing COVID-19-pandemic requires statistical tests for vaccines and treatments, as well as epidemiological test procedures for clusters of infected people. All of these begin with small samples (especially the first and second, until they reach phase II and III in the clinical trials). Thus, bootstrap is still a current issue of research and provides a lot of sub-topics to study.
\clearpage


%%% Bibliographie und Anhang!
\clearpage
\section*{References}
\addcontentsline{toc}{section}{\protect\numberline{}References}%
	\printbibliography[heading = none]
\clearpage
\section*{Appendix A}
\addcontentsline{toc}{section}{\protect\numberline{}Appendix A}%
\begin{table}[ht]
\footnotesize
\centering
\begin{tabular}{rrrrrrrrrrrrr}
  \hline
  & & NP & NP & NP & NP & NP & W & W & W & W & W \\
nboots & n & 0.25 Q & 0.5 Q & Mean & 0.75 Q & Sd & 0.25 Q & 0.5 Q & Mean & 0.75 Q & Sd \\ 
  \hline
10 & 5 & 0.32 & 0.24 & 0.21 & 0.30 & 0.12 & 0.30 & 0.22 & 0.21 & 0.30 & 0.12 \\ 
100 & 5 & 0.28 & 0.22 & 0.20 & 0.29 & 0.12 & 0.31 & 0.23 & 0.23 & 0.31 & 0.12 \\ 
 500 & 5 & 0.30 & 0.24 & 0.23 & 0.30 & 0.11 & 0.27 & 0.20 & 0.19 & 0.28 & 0.11 \\ 
 1000 & 5 & 0.29 & 0.22 & 0.20 & 0.28 & 0.12 & 0.29 & 0.19 & 0.19 & 0.26 & 0.12 \\ 
 10000 & 5 & 0.29 & 0.22 & 0.21 & 0.29 & 0.12 & 0.28 & 0.19 & 0.19 & 0.27 & 0.12 \\ 
  10 & 10 & 0.15 & 0.13 & 0.11 & 0.16 & 0.06 & 0.15 & 0.12 & 0.11 & 0.16 & 0.05 \\ 
 100 & 10 & 0.14 & 0.12 & 0.11 & 0.15 & 0.06 & 0.15 & 0.11 & 0.11 & 0.15 & 0.05 \\ 
 500 & 10 & 0.14 & 0.11 & 0.10 & 0.14 & 0.06 & 0.13 & 0.09 & 0.09 & 0.14 & 0.06 \\ 
 1000 & 10 & 0.15 & 0.12 & 0.10 & 0.14 & 0.06 & 0.14 & 0.10 & 0.10 & 0.13 & 0.05 \\ 
 10000 & 10 & 0.15 & 0.12 & 0.10 & 0.13 & 0.05 & 0.15 & 0.10 & 0.10 & 0.15 & 0.06 \\ 
 10 & 50 & 0.04 & 0.03 & 0.02 & 0.04 & 0.01 & 0.03 & 0.02 & 0.02 & 0.03 & 0.01 \\ 
 100 & 50 & 0.03 & 0.03 & 0.02 & 0.03 & 0.01 & 0.03 & 0.02 & 0.02 & 0.03 & 0.01 \\ 
  500 & 50 & 0.03 & 0.03 & 0.02 & 0.03 & 0.01 & 0.03 & 0.02 & 0.02 & 0.03 & 0.01 \\ 
 1000 & 50 & 0.03 & 0.03 & 0.02 & 0.03 & 0.01 & 0.03 & 0.02 & 0.02 & 0.03 & 0.01 \\ 
 10000 & 50 & 0.03 & 0.03 & 0.02 & 0.03 & 0.01 & 0.03 & 0.02 & 0.02 & 0.03 & 0.01 \\ 
 10 & 100 & 0.02 & 0.02 & 0.01 & 0.02 & 0.01 & 0.02 & 0.01 & 0.01 & 0.02 & 0.01 \\ 
 100 & 100 & 0.02 & 0.01 & 0.01 & 0.02 & 0.00 & 0.01 & 0.01 & 0.01 & 0.02 & 0.01 \\ 
 500 & 100 & 0.02 & 0.01 & 0.01 & 0.02 & 0.01 & 0.02 & 0.01 & 0.01 & 0.02 & 0.00 \\ 
  1000 & 100 & 0.02 & 0.01 & 0.01 & 0.02 & 0.01 & 0.02 & 0.01 & 0.01 & 0.01 & 0.01 \\ 
10000 & 100 & 0.02 & 0.01 & 0.01 & 0.02 & 0.00 & 0.01 & 0.01 & 0.01 & 0.02 & 0.01 \\ 
10 & 200 & 0.01 & 0.01 & 0.01 & 0.01 & 0.00 & 0.01 & 0.01 & 0.01 & 0.01 & 0.00 \\ 
100 & 200 & 0.01 & 0.01 & 0.00 & 0.01 & 0.00 & 0.01 & 0.00 & 0.00 & 0.01 & 0.00 \\ 
500 & 200 & 0.01 & 0.01 & 0.01 & 0.01 & 0.00 & 0.01 & 0.01 & 0.01 & 0.01 & 0.00 \\ 
1000 & 200 & 0.01 & 0.01 & 0.01 & 0.01 & 0.00 & 0.01 & 0.00 & 0.00 & 0.01 & 0.00 \\ 
10000 & 200 & 0.01 & 0.01 & 0.01 & 0.01 & 0.00 & 0.01 & 0.01 & 0.01 & 0.01 & 0.00 \\ 
   \hline
\end{tabular}
\caption[MSE of Summary Statistics R]{MSE of Summary Statistics computed in R. NP stands for the nonparametric bootstrap and W for the wild version. Q stands for quantile, Sd for standard deviation.}
\label{tab:sum_mse_r}
\end{table}

\begin{table}[ht]
\footnotesize
\centering
\begin{tabular}{rrrrrrrrrrrrr}
  \hline
  & & NP & NP & NP & NP & NP & W & W & W & W & W \\
nboots & n & 0.25 Q & 0.5 Q & Mean & 0.75 Q & Sd & 0.25 Q & 0.5 Q & Mean & 0.75 Q & Sd \\ 
  \hline
10 & 5 & 0.45 & 0.39 & 0.36 & 0.43 & 0.28 & 0.44 & 0.37 & 0.37 & 0.45 & 0.28 \\ 
  100 & 5 & 0.43 & 0.37 & 0.35 & 0.43 & 0.29 & 0.45 & 0.38 & 0.39 & 0.44 & 0.28 \\ 
  500 & 5 & 0.44 & 0.39 & 0.38 & 0.44 & 0.28 & 0.42 & 0.35 & 0.35 & 0.42 & 0.28 \\ 
  1000 & 5 & 0.43 & 0.37 & 0.36 & 0.42 & 0.29 & 0.43 & 0.35 & 0.35 & 0.41 & 0.29 \\ 
  10000 & 5 & 0.43 & 0.38 & 0.36 & 0.43 & 0.28 & 0.42 & 0.35 & 0.35 & 0.41 & 0.29 \\ 
  10 & 10 & 0.31 & 0.29 & 0.26 & 0.31 & 0.19 & 0.31 & 0.27 & 0.27 & 0.33 & 0.19 \\ 
  100 & 10 & 0.30 & 0.28 & 0.26 & 0.32 & 0.20 & 0.30 & 0.26 & 0.26 & 0.30 & 0.19 \\ 
  500 & 10 & 0.30 & 0.27 & 0.25 & 0.30 & 0.19 & 0.30 & 0.24 & 0.24 & 0.30 & 0.20 \\ 
  1000 & 10 & 0.31 & 0.28 & 0.26 & 0.30 & 0.19 & 0.30 & 0.25 & 0.25 & 0.29 & 0.18 \\ 
  10000 & 10 & 0.31 & 0.28 & 0.26 & 0.29 & 0.18 & 0.31 & 0.26 & 0.26 & 0.31 & 0.19 \\ 
  10 & 50 & 0.15 & 0.14 & 0.12 & 0.15 & 0.08 & 0.14 & 0.12 & 0.12 & 0.14 & 0.08 \\ 
  100 & 50 & 0.14 & 0.13 & 0.11 & 0.14 & 0.08 & 0.14 & 0.12 & 0.11 & 0.14 & 0.08 \\ 
  500 & 50 & 0.14 & 0.13 & 0.11 & 0.14 & 0.08 & 0.13 & 0.11 & 0.11 & 0.14 & 0.08 \\ 
  1000 & 50 & 0.15 & 0.13 & 0.12 & 0.14 & 0.08 & 0.14 & 0.12 & 0.12 & 0.14 & 0.08 \\ 
  10000 & 50 & 0.14 & 0.13 & 0.11 & 0.14 & 0.08 & 0.14 & 0.11 & 0.11 & 0.14 & 0.09 \\ 
  10 & 100 & 0.11 & 0.10 & 0.08 & 0.11 & 0.06 & 0.10 & 0.09 & 0.08 & 0.10 & 0.06 \\ 
  100 & 100 & 0.10 & 0.09 & 0.08 & 0.10 & 0.06 & 0.10 & 0.08 & 0.08 & 0.10 & 0.06 \\ 
  500 & 100 & 0.10 & 0.10 & 0.08 & 0.10 & 0.06 & 0.10 & 0.08 & 0.08 & 0.10 & 0.06 \\ 
  1000 & 100 & 0.10 & 0.09 & 0.08 & 0.10 & 0.06 & 0.10 & 0.08 & 0.08 & 0.09 & 0.06 \\ 
  10000 & 100 & 0.10 & 0.10 & 0.08 & 0.10 & 0.05 & 0.10 & 0.08 & 0.08 & 0.10 & 0.06 \\ 
  10 & 200 & 0.08 & 0.07 & 0.06 & 0.08 & 0.04 & 0.07 & 0.06 & 0.06 & 0.08 & 0.04 \\ 
  100 & 200 & 0.08 & 0.07 & 0.06 & 0.07 & 0.04 & 0.07 & 0.06 & 0.06 & 0.07 & 0.04 \\ 
  500 & 200 & 0.07 & 0.07 & 0.06 & 0.07 & 0.04 & 0.07 & 0.06 & 0.06 & 0.07 & 0.04 \\ 
  1000 & 200 & 0.08 & 0.07 & 0.06 & 0.07 & 0.04 & 0.07 & 0.06 & 0.06 & 0.07 & 0.04 \\ 
  10000 & 200 & 0.07 & 0.07 & 0.06 & 0.08 & 0.04 & 0.07 & 0.06 & 0.06 & 0.07 & 0.04 \\  
   \hline
\end{tabular}
\caption[MAE of Summary Statistics R]{MAE of Summary Statistics computed in R. NP stands for the nonparametric bootstrap and W for the wild version. Q stands for quantile, Sd for standard deviation.}
\label{tab:sum_mae_r}
\end{table}

\begin{table}[ht]
\footnotesize
\centering
\begin{tabular}{rrrrrrrrrrrrr}
  \hline
  & & NP & NP & NP & NP & NP & W & W & W & W & W \\
nboots & n & 0.25 Q & 0.5 Q & Mean & 0.75 Q & Sd & 0.25 Q & 0.5 Q & Mean & 0.75 Q & Sd \\ 
  \hline
10 & 5 & 0.33 & 0.26 & 0.21 & 0.31 & 0.14 & 0.29 & 0.23 & 0.22 & 0.32 & 0.12 \\ 
  100 & 5 & 0.27 & 0.22 & 0.20 & 0.30 & 0.12 & 0.28 & 0.20 & 0.20 & 0.28 & 0.12 \\ 
  500 & 5 & 0.28 & 0.21 & 0.19 & 0.27 & 0.12 & 0.29 & 0.21 & 0.21 & 0.29 & 0.12 \\ 
  1000 & 5 & 0.29 & 0.21 & 0.20 & 0.26 & 0.12 & 0.26 & 0.20 & 0.20 & 0.27 & 0.11 \\ 
  10000 & 5 & 0.28 & 0.21 & 0.19 & 0.27 & 0.12 & 0.30 & 0.21 & 0.21 & 0.29 & 0.12 \\ 
  10 & 10 & 0.15 & 0.13 & 0.10 & 0.15 & 0.06 & 0.15 & 0.11 & 0.11 & 0.15 & 0.06 \\ 
  100 & 10 & 0.14 & 0.12 & 0.11 & 0.15 & 0.06 & 0.13 & 0.09 & 0.09 & 0.13 & 0.06 \\ 
  500 & 10 & 0.13 & 0.11 & 0.09 & 0.14 & 0.06 & 0.14 & 0.10 & 0.10 & 0.14 & 0.06 \\ 
  1000 & 10 & 0.14 & 0.11 & 0.10 & 0.13 & 0.05 & 0.14 & 0.10 & 0.10 & 0.14 & 0.06 \\ 
  10000 & 10 & 0.14 & 0.12 & 0.10 & 0.14 & 0.06 & 0.14 & 0.10 & 0.10 & 0.14 & 0.05 \\ 
  10 & 50 & 0.03 & 0.03 & 0.02 & 0.03 & 0.01 & 0.03 & 0.02 & 0.02 & 0.03 & 0.01 \\ 
  100 & 50 & 0.03 & 0.03 & 0.02 & 0.03 & 0.01 & 0.03 & 0.02 & 0.02 & 0.03 & 0.01 \\ 
  500 & 50 & 0.03 & 0.03 & 0.02 & 0.03 & 0.01 & 0.03 & 0.02 & 0.02 & 0.03 & 0.01 \\ 
  1000 & 50 & 0.03 & 0.03 & 0.02 & 0.03 & 0.01 & 0.03 & 0.02 & 0.02 & 0.03 & 0.01 \\ 
  10000 & 50 & 0.03 & 0.03 & 0.02 & 0.03 & 0.01 & 0.03 & 0.02 & 0.02 & 0.03 & 0.01 \\ 
  10 & 100 & 0.02 & 0.01 & 0.01 & 0.02 & 0.01 & 0.02 & 0.01 & 0.01 & 0.02 & 0.00 \\ 
  100 & 100 & 0.02 & 0.01 & 0.01 & 0.02 & 0.00 & 0.02 & 0.01 & 0.01 & 0.01 & 0.00 \\ 
  500 & 100 & 0.02 & 0.01 & 0.01 & 0.02 & 0.01 & 0.02 & 0.01 & 0.01 & 0.02 & 0.00 \\ 
  1000 & 100 & 0.02 & 0.01 & 0.01 & 0.02 & 0.01 & 0.02 & 0.01 & 0.01 & 0.02 & 0.01 \\ 
  10000 & 100 & 0.02 & 0.01 & 0.01 & 0.02 & 0.00 & 0.02 & 0.01 & 0.01 & 0.02 & 0.01 \\ 
  10 & 200 & 0.01 & 0.01 & 0.01 & 0.01 & 0.00 & 0.01 & 0.01 & 0.01 & 0.01 & 0.00 \\ 
  100 & 200 & 0.01 & 0.01 & 0.00 & 0.01 & 0.00 & 0.01 & 0.00 & 0.00 & 0.01 & 0.00 \\ 
  500 & 200 & 0.01 & 0.01 & 0.01 & 0.01 & 0.00 & 0.01 & 0.01 & 0.01 & 0.01 & 0.00 \\ 
  1000 & 200 & 0.01 & 0.01 & 0.00 & 0.01 & 0.00 & 0.01 & 0.01 & 0.01 & 0.01 & 0.00 \\ 
  10000 & 200 & 0.01 & 0.01 & 0.00 & 0.01 & 0.00 & 0.01 & 0.01 & 0.01 & 0.01 & 0.00 \\ 
   \hline
\end{tabular}
\caption[MSE of Summary Statistics C++]{MSE of Summary Statistics computed in C++. NP stands for the nonparametric bootstrap and W for the wild version. Q stands for quantile, Sd for standard deviation.}
\label{tab:sum_mse_cpp}
\end{table}

\begin{table}[ht]
\footnotesize
\centering
\begin{tabular}{rrrrrrrrrrrrr}
  \hline
  & & NP & NP & NP & NP & NP & W & W & W & W & W \\
nboots & n & 0.25 Q & 0.5 Q & Mean & 0.75 Q & Sd & 0.25 Q & 0.5 Q & Mean & 0.75 Q & Sd \\ 
  \hline
10 & 5 & 0.47 & 0.41 & 0.37 & 0.45 & 0.31 & 0.43 & 0.38 & 0.38 & 0.46 & 0.28 \\ 
  100 & 5 & 0.42 & 0.37 & 0.36 & 0.44 & 0.29 & 0.43 & 0.36 & 0.36 & 0.43 & 0.29 \\ 
  500 & 5 & 0.42 & 0.37 & 0.35 & 0.41 & 0.29 & 0.43 & 0.36 & 0.36 & 0.43 & 0.28 \\ 
  1000 & 5 & 0.44 & 0.37 & 0.35 & 0.41 & 0.29 & 0.41 & 0.35 & 0.35 & 0.42 & 0.27 \\ 
  10000 & 5 & 0.43 & 0.37 & 0.35 & 0.41 & 0.29 & 0.44 & 0.37 & 0.37 & 0.43 & 0.29 \\ 
  10 & 10 & 0.31 & 0.29 & 0.26 & 0.31 & 0.21 & 0.32 & 0.27 & 0.27 & 0.31 & 0.19 \\ 
  100 & 10 & 0.30 & 0.28 & 0.26 & 0.31 & 0.19 & 0.29 & 0.23 & 0.23 & 0.29 & 0.19 \\ 
  500 & 10 & 0.29 & 0.26 & 0.25 & 0.29 & 0.19 & 0.30 & 0.25 & 0.25 & 0.29 & 0.19 \\ 
  1000 & 10 & 0.29 & 0.27 & 0.25 & 0.29 & 0.19 & 0.30 & 0.25 & 0.25 & 0.30 & 0.20 \\ 
  10000 & 10 & 0.30 & 0.27 & 0.26 & 0.31 & 0.20 & 0.30 & 0.25 & 0.25 & 0.30 & 0.19 \\ 
  10 & 50 & 0.14 & 0.14 & 0.12 & 0.15 & 0.08 & 0.14 & 0.12 & 0.11 & 0.14 & 0.08 \\ 
  100 & 50 & 0.14 & 0.13 & 0.11 & 0.14 & 0.08 & 0.15 & 0.12 & 0.12 & 0.14 & 0.08 \\ 
  500 & 50 & 0.15 & 0.13 & 0.12 & 0.14 & 0.08 & 0.14 & 0.11 & 0.11 & 0.14 & 0.08 \\ 
  1000 & 50 & 0.14 & 0.13 & 0.11 & 0.14 & 0.08 & 0.14 & 0.11 & 0.11 & 0.14 & 0.08 \\ 
  10000 & 50 & 0.15 & 0.13 & 0.11 & 0.14 & 0.08 & 0.14 & 0.12 & 0.12 & 0.14 & 0.08 \\ 
  10 & 100 & 0.11 & 0.10 & 0.08 & 0.11 & 0.06 & 0.10 & 0.08 & 0.08 & 0.10 & 0.05 \\ 
  100 & 100 & 0.10 & 0.10 & 0.08 & 0.10 & 0.05 & 0.10 & 0.08 & 0.08 & 0.10 & 0.06 \\ 
  500 & 100 & 0.10 & 0.09 & 0.07 & 0.10 & 0.06 & 0.10 & 0.08 & 0.08 & 0.10 & 0.06 \\ 
  1000 & 100 & 0.10 & 0.09 & 0.08 & 0.10 & 0.06 & 0.10 & 0.08 & 0.08 & 0.10 & 0.06 \\ 
  10000 & 100 & 0.10 & 0.10 & 0.08 & 0.10 & 0.06 & 0.10 & 0.08 & 0.08 & 0.10 & 0.06 \\ 
  10 & 200 & 0.08 & 0.07 & 0.06 & 0.08 & 0.04 & 0.08 & 0.06 & 0.06 & 0.07 & 0.04 \\ 
  100 & 200 & 0.08 & 0.07 & 0.06 & 0.07 & 0.04 & 0.07 & 0.06 & 0.06 & 0.07 & 0.04 \\ 
  500 & 200 & 0.07 & 0.07 & 0.06 & 0.07 & 0.04 & 0.07 & 0.06 & 0.06 & 0.07 & 0.04 \\ 
  1000 & 200 & 0.07 & 0.07 & 0.05 & 0.07 & 0.04 & 0.07 & 0.06 & 0.06 & 0.07 & 0.04 \\ 
  10000 & 200 & 0.07 & 0.07 & 0.06 & 0.07 & 0.04 & 0.07 & 0.06 & 0.06 & 0.07 & 0.04 \\ 
   \hline
\end{tabular}
\caption[MAE of Summary Statistics C++]{MAE of Summary Statistics computed in C++. NP stands for the nonparametric bootstrap and W for the wild version. Q stands for quantile, Sd for standard deviation.}
\label{tab:sum_mae_cpp}
\end{table}


%\clearpage
%\section*{Appendix B}
%\addcontentsline{toc}{section}{\protect\numberline{}Appendix B}%
%\addtocontents{toc}{\protect\enlargethispage{\baselineskip}}
%\input{appendix_b_1.tex}
%\clearpage
%\input{appendix_b_2.tex}
%\clearpage
%%%%%% Eidesstattliche Erklärung!
%\section*{Declaration of Honesty}
%\input{Honesty.tex}
\end{document}