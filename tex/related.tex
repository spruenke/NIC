In this section we provide an overview of literature relating to the topic of the bootstrap. Important to mention is the article of \citet{boot1} who generalised the jackknife principle to what is the classical bootstrap today. Since then, many advancements, generalizations and applications were conducted by numerous authors. \citet{prob1} provides a general overview of resampling techniques in chapter 13 and \citet{stat1} describe and relate bootstrap methods to Maximum Likelihood, Bayesian Inference and Machine Learning techniques in chapter 8. \\
\\
\noindent \citet{boot4} find that bootstrap and permutation methods for matched pairs give valid asymptotic results even for different distributions which have no treatment effect under the null hypothesis. Their paper relates to estimation of t-type test statistics for paired samples and overcomes the problems of small sample sizes, heterogeneity and different distributions. \citet{boot2} find valid results applying the so-called wild bootstrap method to problems of repeated measurements. The nonparametric bootstrap approach delivers better results than WALD or ANOVA statistics, which are especially useful for small sample sizes. \citet{boot3} find that wild-bootstrapping ranked-based procedures overcome assumptions about normality or homogeneity in general factorial measure designs and deliver asymptotically correct multiple contrast tests. Furthermore, \citet{boot5} find that parametric bootstrap can, under minimal assumptions, overcome the problems of normality assumptions or equal covariance matrices in multivariate factorial designs. \\
\\
\noindent \citet{boot6} and \citet{boot7} describe bootstrap methods for timeseries, which again are used in applied research. For example, \citet{alla} refer to the latter one cominbing portfolios. \citet{erin} use a parametric bootstrap approach to numerically compute optimal CVaR (Conditional Value-at-Risk) portfolios.

