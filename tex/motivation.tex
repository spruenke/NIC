%motivation
Since von Mises provided the statistical definition of probability in terms of relative frequency, the group of frequentist statistics has emerged and dominated the field for many years. However, frequentist statistics rely on large samples and asymptotic results or behavior. This raises the question of how well do such methods work in finite samples and especially small samples. For example, clinical researchers often have to deal with small samples in early studies, due to the missing information on toxicity and biokinetic behavior. Thus, classical frequentist statistics become unreliable. This motivates resampling techniques, to which bootstrap belongs. The idea is to overcome small sample problems by resampling from this sample and obtain artificial large samples by mimicking the distribution of the estimator. Although this paper will only present basic problems, further research will lead to more complex problems, which we will mention in the next section. Specifically, rank and pseudo-rank procedures are of current interest in this field as well as clustered data. For example, the COVID-19-pandemy leads to small clusters of data where multiple (statistical) test problems arise which might be solved by resampling techniques. Furthermore, since R is a popular programming language amongst statisticians we want to adress computational complexity and speed and compare it to lower-level languages, since speed becomes more and more of interest due to more complex problems (such as random forests) and also since bootstrap requires a lot of computations.